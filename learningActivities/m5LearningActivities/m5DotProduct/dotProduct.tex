\documentclass{ximera}
\graphicspath{  %% When looking for images,
{./}            %% look here first,
{./pictures/}   %% then look for a pictures folder,
{../pictures/}  %% which may be a directory up.
{../../pictures/}  %% which may be a directory up.
{../../../pictures/}  %% which may be a directory up.
{../../../../pictures/}  %% which may be a directory up.
}

\usepackage{listings}
\usepackage{circuitikz}
\usepackage{xcolor}
\usepackage{amsmath,amsthm}
\usepackage{subcaption}
\usepackage{graphicx}
\usepackage{tikz}
\usepackage{tikz-3dplot}
\usepackage{amsfonts}
\usepackage{mdframed} % For framing content
\usepackage{tikz-cd}

  \renewcommand{\vector}[1]{\left\langle #1\right\rangle}
  \newcommand{\arrowvec}[1]{{\overset{\rightharpoonup}{#1}}}
  \newcommand{\ro}{\texttt{R}}%% row operation
  \newcommand{\dotp}{\bullet}%% dot product
  \renewcommand{\l}{\ell}
  \let\defaultAnswerFormat\answerFormatBoxed
  \usetikzlibrary{calc,bending}
  \tikzset{>=stealth}
  




%make a maroon color
\definecolor{maroon}{RGB}{128,0,0}
%make a dark blue color
\definecolor{darkblue}{RGB}{0,0,139}
%define the color fourier0 to be the maroon color
\definecolor{fourier0}{RGB}{128,0,0}
%define the color fourier1 to be the dark blue color
\definecolor{fourier1}{RGB}{0,0,139}
%define the color fourier 1t to be the light blue color
\definecolor{fourier1t}{RGB}{173,216,230}
%define the color fourier2 to be the dark green color
\definecolor{fourier2}{RGB}{0,100,0}
%define teh color fourier2t to be the light green color
\definecolor{fourier2t}{RGB}{144,238,144}
%define the color fourier3 to be the dark purple color
\definecolor{fourier3}{RGB}{128,0,128}
%define the color fourier3t to be the light purple color
\definecolor{fourier3t}{RGB}{221,160,221}
%define the color fourier0t to be the red color
\definecolor{fourier0t}{RGB}{255,0,0}
%define the color fourier4 to be the orange color
\definecolor{fourier4}{RGB}{255,165,0}
%define the color fourier4t to be the darker orange color
\definecolor{fourier4t}{RGB}{255,215,0}
%define the color fourier5 to be the yellow color
\definecolor{fourier5}{RGB}{255,255,0}
%define the color fourier5t to be the darker yellow color
\definecolor{fourier5t}{RGB}{255,255,100}
%define the color fourier6 to be the green color
\definecolor{fourier6}{RGB}{0,128,0}
%define the color fourier6t to be the darker green color
\definecolor{fourier6t}{RGB}{0,255,0}

%New commands for this doc for errors in copying
\newcommand{\eigenvar}{\lambda}
%\newcommand{\vect}[1]{\mathbf{#1}}
\renewcommand{\th}{^{\text{th}}}
\newcommand{\st}{^{\text{st}}}
\newcommand{\nd}{^{\text{nd}}}
\newcommand{\rd}{^{\text{rd}}}
\newcommand{\paren}[1]{\left(#1\right)}
\newcommand{\abs}[1]{\left|#1\right|}
\newcommand{\R}{\mathbb{R}}
\newcommand{\C}{\mathbb{C}}
\newcommand{\Hilb}{\mathbb{H}}
\newcommand{\qq}[1]{\text{#1}}
\newcommand{\Z}{\mathbb{Z}}
\newcommand{\N}{\mathbb{N}}
\newcommand{\q}[1]{\text{``#1''}}
%\newcommand{\mat}[1]{\begin{bmatrix}#1\end{bmatrix}}
\newcommand{\rref}{\text{reduced row echelon form}}
\newcommand{\ef}{\text{echelon form}}
\newcommand{\ohm}{\Omega}
\newcommand{\volt}{\text{V}}
\newcommand{\amp}{\text{A}}
\newcommand{\Seq}{\textbf{Seq}}
\newcommand{\Poly}{\textbf{P}}
\renewcommand{\quad}{\text{    }}
\newcommand{\roweq}{\simeq}
\newcommand{\rowop}{\simeq}
\newcommand{\rowswap}{\leftrightarrow}
\newcommand{\Mat}{\textbf{M}}
\newcommand{\Func}{\textbf{Func}}
\newcommand{\Hw}{\textbf{Hamming weight}}
\newcommand{\Hd}{\textbf{Hamming distance}}
\newcommand{\rank}{\text{rank}}
\newcommand{\longvect}[1]{\overrightarrow{#1}}
% Define the circled command
\newcommand{\circled}[1]{%
  \tikz[baseline=(char.base)]{
    \node[shape=circle,draw,inner sep=2pt,red,fill=red!20,text=black] (char) {#1};}%
}

% Define custom command \strikeh that just puts red text on the 2nd argument
\newcommand{\strikeh}[2]{\textcolor{red}{#2}}

% Define custom command \strikev that just puts red text on the 2nd argument
\newcommand{\strikev}[2]{\textcolor{red}{#2}}

%more new commands for this doc for errors in copying
\newcommand{\SI}{\text{SI}}
\newcommand{\kg}{\text{kg}}
\newcommand{\m}{\text{m}}
\newcommand{\s}{\text{s}}
\newcommand{\norm}[1]{\left\|#1\right\|}
\newcommand{\col}{\text{col}}
\newcommand{\sspan}{\text{span}}
\newcommand{\proj}{\text{proj}}
\newcommand{\set}[1]{\left\{#1\right\}}
\newcommand{\degC}{^\circ\text{C}}
\newcommand{\centroid}[1]{\overline{#1}}
\newcommand{\dotprod}{\boldsymbol{\cdot}}
%\newcommand{\coord}[1]{\begin{bmatrix}#1\end{bmatrix}}
\newcommand{\iprod}[1]{\langle #1 \rangle}
\newcommand{\adjoint}{^{*}}
\newcommand{\conjugate}[1]{\overline{#1}}
\newcommand{\eigenvarA}{\lambda}
\newcommand{\eigenvarB}{\mu}
\newcommand{\orth}{\perp}
\newcommand{\bigbracket}[1]{\left[#1\right]}
\newcommand{\textiff}{\text{ if and only if }}
\newcommand{\adj}{\text{adj}}
\newcommand{\ijth}{\emph{ij}^\text{th}}
\newcommand{\minor}[2]{M_{#2}}
\newcommand{\cofactor}{\text{C}}
\newcommand{\shift}{\textbf{shift}}
\newcommand{\startmat}[1]{
  \left[\begin{array}{#1}
}
\newcommand{\stopmat}{\end{array}\right]}
%a command to give a name to explorations and hints and theorems
\newcommand{\name}[1]{\begin{centering}\textbf{#1}\end{centering}}
\newcommand{\vect}[1]{\vec{#1}}
\newcommand{\dfn}[1]{\textbf{#1}}
\newcommand{\transpose}{\mathsf{T}}
\newcommand{\mtlb}[2][black]{\texttt{\textcolor{#1}{#2}}}
\newcommand{\RR}{\mathbb{R}} % Real numbers
\newcommand{\id}{\text{id}}

\author{Zack Reed}
%borrowed from anna davis
\title{Learning Activity: }
\begin{document}
\begin{abstract}

    In this learning activity, you will be 
\end{abstract}
\maketitle

\section*{Dot Product and its Properties}
 
\begin{definition}\label{def:dotproduct}
  Let $\vec{u}$ and $\vec{v}$ be vectors in $\RR^n$.  The \dfn{dot
    product} of $\vec{u}$ and $\vec{v}$, denoted by
  $\vec{u}\dotp \vec{v}$, is given by
  \begin{align*}
    \vec{u}\dotp\vec{v}=\begin{bmatrix}u_1\\u_2\\\vdots\\u_n\end{bmatrix}\dotp\begin{bmatrix}v_1\\v_2\\\vdots\\v_n\end{bmatrix}=u_1v_1+u_2v_2+\ldots+u_nv_n.
  \end{align*}
\end{definition}
 
\begin{example}\label{ex:dotex}
  Find $\vec{u}\dotp \vec{v}$ if
  $\vec{u}=\begin{bmatrix}-2\\0\\1\end{bmatrix}$ and
  $\vec{v}=\begin{bmatrix}3\\2\\-4\end{bmatrix}$.
 
  \begin{explanation}
    $$\vec{u}\dotp\vec{v}=\begin{bmatrix}-2\\0\\1\end{bmatrix}\dotp\begin{bmatrix}3\\2\\-4\end{bmatrix}=(-2)(3)+(0)(2)+(1)(-4)=-6-4=-10$$
  \end{explanation}
\end{example}
 
Note that the dot product of two vectors is a scalar.  For this reason, the dot product is sometimes called a \dfn{scalar product}.
 
\subsection*{Properties of the Dot Product}
 
A quick examination of Example \ref{ex:dotex} will convince you that the dot product is \dfn{commutative}. In other words, $\vec{u}\dotp\vec{v}=\vec{v}\dotp\vec{u}$.  This and other properties of the dot product are stated below.
 
\begin{theorem}\label{th:dotproductproperties} The following properties hold for
  vectors $\vec{u}$, $\vec{v}$ and $\vec{w}$ in $\RR^n$ and scalar
  $k$ in $\RR$.
  \begin{enumerate}
  \item\label{item:commutative}
    $\vec{u}\dotp\vec{v}=\vec{v}\dotp\vec{u}$
    
  \item\label{item:distributive} $(\vec{u}+\vec{v})\dotp \vec{w}=\vec{u}\dotp \vec{w}+\vec{v}\dotp \vec{w}$
    
  \item\label{item:distributive-again} $\vec{u}\dotp (\vec{v}+\vec{w})=\vec{u}\dotp\vec{v}+\vec{u}\dotp \vec{w}$
    
  \item\label{item:scalar} $(k\vec{u})\dotp \vec{v}=k(\vec{u}\dotp\vec{v})=\vec{u}\dotp (k\vec{v})$
    
  \item \label{item:positive} $\vec{u}\dotp\vec{u}\geq 0$, and $\vec{u}\dotp\vec{u}=0$ if and only if $\vec{u}={\bf 0}$.
    
  \item \label{item:norm}
    $\norm{\vec{u}}^2=\vec{u}\dotp\vec{u}$
  \end{enumerate}
\end{theorem}
 
We will prove Property ~\ref{item:distributive}.  The remaining properties are left as exercises.
 
\begin{proof}[Proof of Property~\ref{item:distributive}:]
 
\begin{align*}
\left(\vec{u}+\vec{v}\right)\dotp \vec{w}&=\left(\begin{bmatrix} u_1\\ u_2\\ \vdots\\ u_n \end{bmatrix}+\begin{bmatrix} v_1\\ v_2\\ \vdots\\ v_n \end{bmatrix}\right)\dotp \begin{bmatrix}w_1\\w_2\\\vdots\\w_n\end{bmatrix}=\begin{bmatrix}
u_1+v_1\\
u_2+v_2\\
\vdots\\
u_n+v_n
\end{bmatrix}\dotp \begin{bmatrix}w_1\\w_2\\\vdots\\w_n\end{bmatrix}\\
&=(u_1+v_1)w_1+
(u_2+v_2)w_2+
\ldots+
(u_n+v_n)w_n\\
&=u_1w_1+v_1w_1+
u_2w_2+v_2w_2+
\ldots+
u_nw_n+v_nw_n\\
&=(u_1w_1+
u_2w_2\ldots+u_nw_n)+(v_1w_1+v_2w_2+
\ldots
+v_nw_n)\\
&=\begin{bmatrix}
u_1\\
u_2\\
\vdots\\
u_n
\end{bmatrix}\dotp\begin{bmatrix}w_1\\w_2\\\vdots\\w_n\end{bmatrix}+\begin{bmatrix}
v_1\\
v_2\\
\vdots\\
v_n
\end{bmatrix}\dotp \begin{bmatrix}w_1\\w_2\\\vdots\\w_n\end{bmatrix}
=\vec{u}\dotp\vec{w}+\vec{v}\dotp\vec{w}
\end{align*}
\end{proof}
 
We will illustrate Property~\ref{item:norm} with an example.
\begin{example}\label{ex:exprop6}
  Let $\vec{u}=\begin{bmatrix}-2\\3\end{bmatrix}$.  Use $\vec{u}$ to illustrate Property~\ref{item:norm} of Theorem~\ref{th:dotproductproperties}.
  \begin{explanation}
   
  $$\norm{\vec{u}}^2=(-2)^2+3^2=(-2)(-2)+(3)(3)=\begin{bmatrix}-2\\3\end{bmatrix}\dotp\begin{bmatrix}-2\\3\end{bmatrix}=\vec{u}\dotp\vec{u}$$
  \end{explanation}
\end{example}
 
If we take the square root of both sides of the equation in Property~\ref{item:norm}, we get an alternative way to think of the \href{https://ximera.osu.edu/oerlinalg/LinearAlgebra/VEC-0020/main}{length of a vector}.
 
\begin{corollary}[Length of a Vector]\label{cor:length_via_dotprod}
    Let $\vec{v}$ be a vector in $\RR^n$, then the \dfn{length}, or the \dfn{magnitude}, of $\vec{v}$ is given by
\begin{equation*} \label{eq:norm_dotp}
\norm{\vec{v}}=\sqrt{\vec{v} \dotp \vec{v}}
\end{equation*}
\end{corollary}
 
\subsection*{Practice Problems}
 
\begin{problem}\label{prob:dotproduct1}
Find the dot product of $\vec{u}$ and $\vec{v}$ if
  $$\vec{u}=\begin{bmatrix}-1\\-2\\5\\4\end{bmatrix},\quad \vec{v}=\begin{bmatrix}2\\-2\\-3\\1\end{bmatrix}$$
  Answer:
  $$\vec{u} \dotp \vec{v} = \answer{-9}$$
\end{problem}
 
\begin{problem}\label{prob:dotproduct2}
Find the dot product of $\vec{u}$ and $\vec{v}$ if
  $$\vec{u}=\begin{bmatrix}1\\1/2\end{bmatrix},\quad \vec{v}=\begin{bmatrix}-2\\4\end{bmatrix}$$
  Answer:
   
  $$\vec{u} \dotp \vec{v} = \answer{0}$$
\end{problem}
 
\begin{problem}\label{prob:dotproductprop6}
  Use vector $\vec{u}=\begin{bmatrix}2\\5\\-7\end{bmatrix}$ to
  illustrate Property~\ref{item:norm} of Theorem~\ref{th:dotproductproperties}.
\end{problem}
 
\begin{problem}\label{prob:th:dotprductproperties}
  Prove Properties~\ref{item:commutative}, \ref{item:distributive-again}, \ref{item:scalar}, \ref{item:positive} and \ref{item:norm} of Theorem~\ref{th:dotproductproperties}.
\end{problem}
 
\begin{problem}\label{prob:perpvectors1}
From the given list of vector pairs, identify ALL pairs of vectors that lie on perpendicular lines.
\begin{hint}
You may want to draw a picture and think about what you know about slopes of perpendicular lines.
\end{hint}
\begin{selectAll}
  \choice[correct]{$\vec{u}=\begin{bmatrix}1\\\frac{1}{2}\end{bmatrix}$, $\vec{v}=\begin{bmatrix}-2\\4\end{bmatrix}$}
  \choice{$\vec{u}=\begin{bmatrix}-1\\\frac{1}{2}\end{bmatrix}$, $\vec{v}=\begin{bmatrix}-2\\4\end{bmatrix}$}
  \choice[correct]{$\vec{u}=\begin{bmatrix}1\\\frac{1}{2}\end{bmatrix}$, $\vec{v}=\begin{bmatrix}1\\-2\end{bmatrix}$}
  \choice[correct]{$\vec{u}=\begin{bmatrix}-1\\-\frac{1}{2}\end{bmatrix}$, $\vec{v}=\begin{bmatrix}-2\\4\end{bmatrix}$}
\end{selectAll}
Compute $\vec{u}\dotp\vec{v}$ for each pair.  What do you observe?
\end{problem}
 
\begin{problem}%\label{prob:perpvectors2}
 For each problem below
 \begin{enumerate}
 \item
 Find the value of $x$ that will make vectors $\vec{u}$ and $\vec{v}$ perpendicular.
  \begin{hint} Think of the $x$-component as the ``run" and the $y$-component as the ``rise", then use what you know about slopes of perpendicular lines.
  \end{hint}
  \item Find $\vec{u}\dotp\vec{v}$.
  \end{enumerate}
   
  
  \begin{problem}\label{prob:perpvectors2a}
    $$\vec{u} = \begin{bmatrix}1\\2\end{bmatrix},\quad \vec{v}=\begin{bmatrix}2\\x\end{bmatrix}$$
     
    Answer: 
    $$x = \answer{-1}$$
    $$\vec{u}\dotp\vec{v}=\answer{0}$$
  \end{problem}
 
  \begin{problem}\label{prob:perpvectors2b}
    $$\vec{u} = \begin{bmatrix}5\\2\end{bmatrix},\quad \vec{v}=\begin{bmatrix}x\\-4\end{bmatrix}$$
    Answer:
    $$x = \answer{8/5}$$
    $$\vec{u}\dotp\vec{v}=\answer{0}$$
  \end{problem}
 
  \begin{problem}\label{prob:perpvectors2c}
    $$\vec{u} = \begin{bmatrix} 4\\-3\end{bmatrix},\quad \vec{v} =\begin{bmatrix}6\\x\end{bmatrix}$$
    Answer:
    $$x = \answer{8}$$
    $$\vec{u}\dotp\vec{v}=\answer{0}$$
  \end{problem}
\end{problem}
 
\begin{problem}\label{prob:perpvectors3}
  \begin{enumerate}
    \item Vector $\vec{u}$ that lies on the line $y=mx$, has the form $\vec{u}=k\begin{bmatrix}1\\m\end{bmatrix}$.  Assuming that $m\neq 0$, find the general form for a vector $\vec{v}$ that lies on a line perpendicular to $y=mx$.
      \begin{hint}
        What do you know about the slopes of perpendicular
        lines?
      \end{hint}
    \item Find $\vec{u}\dotp \vec{v}$.
    \item Formulate a conjecture about the dot product of perpendicular vectors.
  \end{enumerate}
\end{problem}


\end{document}