\documentclass{ximera}
\graphicspath{  %% When looking for images,
{./}            %% look here first,
{./pictures/}   %% then look for a pictures folder,
{../pictures/}  %% which may be a directory up.
{../../pictures/}  %% which may be a directory up.
{../../../pictures/}  %% which may be a directory up.
{../../../../pictures/}  %% which may be a directory up.
}

\usepackage{listings}
\usepackage{circuitikz}
\usepackage{xcolor}
\usepackage{amsmath,amsthm}
\usepackage{subcaption}
\usepackage{graphicx}
\usepackage{tikz}
\usepackage{tikz-3dplot}
\usepackage{amsfonts}
\usepackage{mdframed} % For framing content
\usepackage{tikz-cd}

  \renewcommand{\vector}[1]{\left\langle #1\right\rangle}
  \newcommand{\arrowvec}[1]{{\overset{\rightharpoonup}{#1}}}
  \newcommand{\ro}{\texttt{R}}%% row operation
  \newcommand{\dotp}{\bullet}%% dot product
  \renewcommand{\l}{\ell}
  \let\defaultAnswerFormat\answerFormatBoxed
  \usetikzlibrary{calc,bending}
  \tikzset{>=stealth}
  




%make a maroon color
\definecolor{maroon}{RGB}{128,0,0}
%make a dark blue color
\definecolor{darkblue}{RGB}{0,0,139}
%define the color fourier0 to be the maroon color
\definecolor{fourier0}{RGB}{128,0,0}
%define the color fourier1 to be the dark blue color
\definecolor{fourier1}{RGB}{0,0,139}
%define the color fourier 1t to be the light blue color
\definecolor{fourier1t}{RGB}{173,216,230}
%define the color fourier2 to be the dark green color
\definecolor{fourier2}{RGB}{0,100,0}
%define teh color fourier2t to be the light green color
\definecolor{fourier2t}{RGB}{144,238,144}
%define the color fourier3 to be the dark purple color
\definecolor{fourier3}{RGB}{128,0,128}
%define the color fourier3t to be the light purple color
\definecolor{fourier3t}{RGB}{221,160,221}
%define the color fourier0t to be the red color
\definecolor{fourier0t}{RGB}{255,0,0}
%define the color fourier4 to be the orange color
\definecolor{fourier4}{RGB}{255,165,0}
%define the color fourier4t to be the darker orange color
\definecolor{fourier4t}{RGB}{255,215,0}
%define the color fourier5 to be the yellow color
\definecolor{fourier5}{RGB}{255,255,0}
%define the color fourier5t to be the darker yellow color
\definecolor{fourier5t}{RGB}{255,255,100}
%define the color fourier6 to be the green color
\definecolor{fourier6}{RGB}{0,128,0}
%define the color fourier6t to be the darker green color
\definecolor{fourier6t}{RGB}{0,255,0}

%New commands for this doc for errors in copying
\newcommand{\eigenvar}{\lambda}
%\newcommand{\vect}[1]{\mathbf{#1}}
\renewcommand{\th}{^{\text{th}}}
\newcommand{\st}{^{\text{st}}}
\newcommand{\nd}{^{\text{nd}}}
\newcommand{\rd}{^{\text{rd}}}
\newcommand{\paren}[1]{\left(#1\right)}
\newcommand{\abs}[1]{\left|#1\right|}
\newcommand{\R}{\mathbb{R}}
\newcommand{\C}{\mathbb{C}}
\newcommand{\Hilb}{\mathbb{H}}
\newcommand{\qq}[1]{\text{#1}}
\newcommand{\Z}{\mathbb{Z}}
\newcommand{\N}{\mathbb{N}}
\newcommand{\q}[1]{\text{``#1''}}
%\newcommand{\mat}[1]{\begin{bmatrix}#1\end{bmatrix}}
\newcommand{\rref}{\text{reduced row echelon form}}
\newcommand{\ef}{\text{echelon form}}
\newcommand{\ohm}{\Omega}
\newcommand{\volt}{\text{V}}
\newcommand{\amp}{\text{A}}
\newcommand{\Seq}{\textbf{Seq}}
\newcommand{\Poly}{\textbf{P}}
\renewcommand{\quad}{\text{    }}
\newcommand{\roweq}{\simeq}
\newcommand{\rowop}{\simeq}
\newcommand{\rowswap}{\leftrightarrow}
\newcommand{\Mat}{\textbf{M}}
\newcommand{\Func}{\textbf{Func}}
\newcommand{\Hw}{\textbf{Hamming weight}}
\newcommand{\Hd}{\textbf{Hamming distance}}
\newcommand{\rank}{\text{rank}}
\newcommand{\longvect}[1]{\overrightarrow{#1}}
% Define the circled command
\newcommand{\circled}[1]{%
  \tikz[baseline=(char.base)]{
    \node[shape=circle,draw,inner sep=2pt,red,fill=red!20,text=black] (char) {#1};}%
}

% Define custom command \strikeh that just puts red text on the 2nd argument
\newcommand{\strikeh}[2]{\textcolor{red}{#2}}

% Define custom command \strikev that just puts red text on the 2nd argument
\newcommand{\strikev}[2]{\textcolor{red}{#2}}

%more new commands for this doc for errors in copying
\newcommand{\SI}{\text{SI}}
\newcommand{\kg}{\text{kg}}
\newcommand{\m}{\text{m}}
\newcommand{\s}{\text{s}}
\newcommand{\norm}[1]{\left\|#1\right\|}
\newcommand{\col}{\text{col}}
\newcommand{\sspan}{\text{span}}
\newcommand{\proj}{\text{proj}}
\newcommand{\set}[1]{\left\{#1\right\}}
\newcommand{\degC}{^\circ\text{C}}
\newcommand{\centroid}[1]{\overline{#1}}
\newcommand{\dotprod}{\boldsymbol{\cdot}}
%\newcommand{\coord}[1]{\begin{bmatrix}#1\end{bmatrix}}
\newcommand{\iprod}[1]{\langle #1 \rangle}
\newcommand{\adjoint}{^{*}}
\newcommand{\conjugate}[1]{\overline{#1}}
\newcommand{\eigenvarA}{\lambda}
\newcommand{\eigenvarB}{\mu}
\newcommand{\orth}{\perp}
\newcommand{\bigbracket}[1]{\left[#1\right]}
\newcommand{\textiff}{\text{ if and only if }}
\newcommand{\adj}{\text{adj}}
\newcommand{\ijth}{\emph{ij}^\text{th}}
\newcommand{\minor}[2]{M_{#2}}
\newcommand{\cofactor}{\text{C}}
\newcommand{\shift}{\textbf{shift}}
\newcommand{\startmat}[1]{
  \left[\begin{array}{#1}
}
\newcommand{\stopmat}{\end{array}\right]}
%a command to give a name to explorations and hints and theorems
\newcommand{\name}[1]{\begin{centering}\textbf{#1}\end{centering}}
\newcommand{\vect}[1]{\vec{#1}}
\newcommand{\dfn}[1]{\textbf{#1}}
\newcommand{\transpose}{\mathsf{T}}
\newcommand{\mtlb}[2][black]{\texttt{\textcolor{#1}{#2}}}
\newcommand{\RR}{\mathbb{R}} % Real numbers
\newcommand{\id}{\text{id}}

\author{Zack Reed} %borrowed from Bart and PEter Selinger
\title{Learning Activity: Matrix Properties and Operations}
\begin{document}
\begin{abstract}
Here we introduce matrices similar to vectors
\end{abstract}
\maketitle
 
\section*{Matrix Properties and Operations}

\begin{remark}

  Because it is productive to think about matrices as arrays of vectors, we can take linear combinations of matrices in the same way we do with vectors. This is a powerful tool for data manipulation and analysis, and we'll gain even more useful relationships between matrices and vectors in the next chapter.

  For now, some definitions to make more precise the idea of a linear combination of matrices:

  \begin{definition}\name{Addition of matrices}
    Let $A$ and $B$ be two
    $m\times n$-matrices. Then $A+B=C$%
    \index{matrix!addition}%
    \index{sum|see{addition}}%
    \index{addition!of matrices} where $C$ is the $m\times n$-matrix
    $C$ defined by
    \begin{equation*}
      C_{ij}=A_{ij}+B_{ij}
    \end{equation*}

    Said differently, you add matrices by adding their corresponding entries.

    (IMPORTANT NOTE: You can only add matrices of the same size. If you have missing data, you need to make a decision on how to fill it in before you can add matrices together.)
  \end{definition}

  \begin{definition}\name{Scalar multiplication of matrices}
    Let $A$ be an $m\times n$-matrix and let $c$ be a scalar. Then $cA=D$%
    \index{matrix!scalar multiplication}%
    \index{scalar multiplication!of matrices} where $D$ is the $m\times n$-matrix
    $D$ defined by
    \begin{equation*}
      D_{ij}=cA_{ij}
    \end{equation*}

    Said differently, you multiply a matrix by a scalar by multiplying each entry of the matrix by the scalar.
  \end{definition}

\end{remark}

Let's finish by practicing on some basic matrices. If possible, perform the specified linear combinations.

\[
A = \begin{pmatrix}
2 & -1 & 0 \\
4 & 3 & -2 \\
\end{pmatrix}, \quad
B = \begin{pmatrix}
0 & 2 \\
-1 & 3
\end{pmatrix}, \quad
C = \begin{pmatrix}
4 & 0 & 1 \\
-2 & 2 & 5
\end{pmatrix}, \quad
D = \begin{pmatrix}
1 & -3 \\
2 & 1 \\
0 & 4
\end{pmatrix}
\]

\begin{enumerate}
\item Find $2A+3B$.
\begin{solution}

  $2A+3B$ is \wordChoice{
    \choice{possible}
    \choice[correct]{not possible}
  } because the matrices \wordChoice{
    \choice[correct]{are not}
    \choice{are}
  } the same size.

\end{solution}
\item Find $-2C+5D$.

\begin{solution}

  $-2C+5D$ is \wordChoice{
    \choice{possible}
    \choice[correct]{not possible}
  } because the matrices \wordChoice{
    \choice[correct]{are not}
    \choice{are}
  } the same size.

\begin{problem}

  If we instead took the transpose of $C$, then 
  
  $$-2C^T+5D=\begin{pmatrix} \answer{-3} & -11 \\ 10 & \answer{1} \\ \answer{-2} & 10 \end{pmatrix}.$$

\end{problem}

\end{solution}

\item Find $3A+5C-1D$.

\begin{solution}

  $3A+5C-1D$ is \wordChoice{
    \choice{possible}
    \choice[correct]{not possible}
  } because the matrices \wordChoice{
    \choice{are}
    \choice[correct]{are not}
  } the same size.

  \begin{problem}

    If we instead took the transpose of $D$, then 
    
    $$3A+5C-1D^T=\begin{pmatrix} 25 & \answer{-5} & \answer{5} \\ 5 & \answer{18} & 15 \end{pmatrix}.$$

  \end{problem}

\end{solution}

\end{enumerate}


\end{document}