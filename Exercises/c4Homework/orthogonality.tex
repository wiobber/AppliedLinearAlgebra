\documentclass{ximera}
\graphicspath{  %% When looking for images,
{./}            %% look here first,
{./pictures/}   %% then look for a pictures folder,
{../pictures/}  %% which may be a directory up.
{../../pictures/}  %% which may be a directory up.
{../../../pictures/}  %% which may be a directory up.
{../../../../pictures/}  %% which may be a directory up.
}

\usepackage{listings}
\usepackage{circuitikz}
\usepackage{xcolor}
\usepackage{amsmath,amsthm}
\usepackage{subcaption}
\usepackage{graphicx}
\usepackage{tikz}
\usepackage{tikz-3dplot}
\usepackage{amsfonts}
\usepackage{mdframed} % For framing content
\usepackage{tikz-cd}

  \renewcommand{\vector}[1]{\left\langle #1\right\rangle}
  \newcommand{\arrowvec}[1]{{\overset{\rightharpoonup}{#1}}}
  \newcommand{\ro}{\texttt{R}}%% row operation
  \newcommand{\dotp}{\bullet}%% dot product
  \renewcommand{\l}{\ell}
  \let\defaultAnswerFormat\answerFormatBoxed
  \usetikzlibrary{calc,bending}
  \tikzset{>=stealth}
  




%make a maroon color
\definecolor{maroon}{RGB}{128,0,0}
%make a dark blue color
\definecolor{darkblue}{RGB}{0,0,139}
%define the color fourier0 to be the maroon color
\definecolor{fourier0}{RGB}{128,0,0}
%define the color fourier1 to be the dark blue color
\definecolor{fourier1}{RGB}{0,0,139}
%define the color fourier 1t to be the light blue color
\definecolor{fourier1t}{RGB}{173,216,230}
%define the color fourier2 to be the dark green color
\definecolor{fourier2}{RGB}{0,100,0}
%define teh color fourier2t to be the light green color
\definecolor{fourier2t}{RGB}{144,238,144}
%define the color fourier3 to be the dark purple color
\definecolor{fourier3}{RGB}{128,0,128}
%define the color fourier3t to be the light purple color
\definecolor{fourier3t}{RGB}{221,160,221}
%define the color fourier0t to be the red color
\definecolor{fourier0t}{RGB}{255,0,0}
%define the color fourier4 to be the orange color
\definecolor{fourier4}{RGB}{255,165,0}
%define the color fourier4t to be the darker orange color
\definecolor{fourier4t}{RGB}{255,215,0}
%define the color fourier5 to be the yellow color
\definecolor{fourier5}{RGB}{255,255,0}
%define the color fourier5t to be the darker yellow color
\definecolor{fourier5t}{RGB}{255,255,100}
%define the color fourier6 to be the green color
\definecolor{fourier6}{RGB}{0,128,0}
%define the color fourier6t to be the darker green color
\definecolor{fourier6t}{RGB}{0,255,0}

%New commands for this doc for errors in copying
\newcommand{\eigenvar}{\lambda}
%\newcommand{\vect}[1]{\mathbf{#1}}
\renewcommand{\th}{^{\text{th}}}
\newcommand{\st}{^{\text{st}}}
\newcommand{\nd}{^{\text{nd}}}
\newcommand{\rd}{^{\text{rd}}}
\newcommand{\paren}[1]{\left(#1\right)}
\newcommand{\abs}[1]{\left|#1\right|}
\newcommand{\R}{\mathbb{R}}
\newcommand{\C}{\mathbb{C}}
\newcommand{\Hilb}{\mathbb{H}}
\newcommand{\qq}[1]{\text{#1}}
\newcommand{\Z}{\mathbb{Z}}
\newcommand{\N}{\mathbb{N}}
\newcommand{\q}[1]{\text{``#1''}}
%\newcommand{\mat}[1]{\begin{bmatrix}#1\end{bmatrix}}
\newcommand{\rref}{\text{reduced row echelon form}}
\newcommand{\ef}{\text{echelon form}}
\newcommand{\ohm}{\Omega}
\newcommand{\volt}{\text{V}}
\newcommand{\amp}{\text{A}}
\newcommand{\Seq}{\textbf{Seq}}
\newcommand{\Poly}{\textbf{P}}
\renewcommand{\quad}{\text{    }}
\newcommand{\roweq}{\simeq}
\newcommand{\rowop}{\simeq}
\newcommand{\rowswap}{\leftrightarrow}
\newcommand{\Mat}{\textbf{M}}
\newcommand{\Func}{\textbf{Func}}
\newcommand{\Hw}{\textbf{Hamming weight}}
\newcommand{\Hd}{\textbf{Hamming distance}}
\newcommand{\rank}{\text{rank}}
\newcommand{\longvect}[1]{\overrightarrow{#1}}
% Define the circled command
\newcommand{\circled}[1]{%
  \tikz[baseline=(char.base)]{
    \node[shape=circle,draw,inner sep=2pt,red,fill=red!20,text=black] (char) {#1};}%
}

% Define custom command \strikeh that just puts red text on the 2nd argument
\newcommand{\strikeh}[2]{\textcolor{red}{#2}}

% Define custom command \strikev that just puts red text on the 2nd argument
\newcommand{\strikev}[2]{\textcolor{red}{#2}}

%more new commands for this doc for errors in copying
\newcommand{\SI}{\text{SI}}
\newcommand{\kg}{\text{kg}}
\newcommand{\m}{\text{m}}
\newcommand{\s}{\text{s}}
\newcommand{\norm}[1]{\left\|#1\right\|}
\newcommand{\col}{\text{col}}
\newcommand{\sspan}{\text{span}}
\newcommand{\proj}{\text{proj}}
\newcommand{\set}[1]{\left\{#1\right\}}
\newcommand{\degC}{^\circ\text{C}}
\newcommand{\centroid}[1]{\overline{#1}}
\newcommand{\dotprod}{\boldsymbol{\cdot}}
%\newcommand{\coord}[1]{\begin{bmatrix}#1\end{bmatrix}}
\newcommand{\iprod}[1]{\langle #1 \rangle}
\newcommand{\adjoint}{^{*}}
\newcommand{\conjugate}[1]{\overline{#1}}
\newcommand{\eigenvarA}{\lambda}
\newcommand{\eigenvarB}{\mu}
\newcommand{\orth}{\perp}
\newcommand{\bigbracket}[1]{\left[#1\right]}
\newcommand{\textiff}{\text{ if and only if }}
\newcommand{\adj}{\text{adj}}
\newcommand{\ijth}{\emph{ij}^\text{th}}
\newcommand{\minor}[2]{M_{#2}}
\newcommand{\cofactor}{\text{C}}
\newcommand{\shift}{\textbf{shift}}
\newcommand{\startmat}[1]{
  \left[\begin{array}{#1}
}
\newcommand{\stopmat}{\end{array}\right]}
%a command to give a name to explorations and hints and theorems
\newcommand{\name}[1]{\begin{centering}\textbf{#1}\end{centering}}
\newcommand{\vect}[1]{\vec{#1}}
\newcommand{\dfn}[1]{\textbf{#1}}
\newcommand{\transpose}{\mathsf{T}}
\newcommand{\mtlb}[2][black]{\texttt{\textcolor{#1}{#2}}}
\newcommand{\RR}{\mathbb{R}} % Real numbers
\newcommand{\id}{\text{id}}

\author{Zack Reed}
%borrowed from selinger linear algebra
\begin{document}

\section*{Exercises}

\begin{exercise}
  Let $A=\startmat{ccc} 1 & 1 & 0 \\ 1 & 2 & 0 \\ 0 & 0 & 2 \stopmat$, and consider $\R^3$ with the inner product given by
  $\iprod{\vect{u},\vect{v}} = \vect{u}^TA\vect{v}$. Which of the
  following vectors are orthogonal to each other?
  \begin{equation*}
    \vect{u}_1 = \startmat{c}  1 \\  1 \\  1 \stopmat,\quad
    \vect{u}_2 = \startmat{c} -1 \\  2 \\ -2 \stopmat,\quad
    \vect{u}_3 = \startmat{c}  7 \\ -5 \\ -2 \stopmat,\quad
    \vect{u}_4 = \startmat{c} 10 \\ -2 \\ -7 \stopmat.
  \end{equation*}
  \begin{solution}
    We have
    $\iprod{\vect{u}_1,\vect{u}_2} = \vect{u}_1^T A \vect{u}_2 = 0$,
    $\iprod{\vect{u}_1,\vect{u}_3} = \vect{u}_1^T A \vect{u}_3 = -5$,
    $\iprod{\vect{u}_1,\vect{u}_4} = \vect{u}_1^T A \vect{u}_4 = 0$,
    $\iprod{\vect{u}_2,\vect{u}_3} = \vect{u}_2^T A \vect{u}_3 = 0$,
    $\iprod{\vect{u}_2,\vect{u}_4} = \vect{u}_2^T A \vect{u}_4 = 32$,
    and
    $\iprod{\vect{u}_3,\vect{u}_4} = \vect{u}_3^T A \vect{u}_4 = 54$.
    Therefore, $\vect{u}_1\orth\vect{u}_2$,
    $\vect{u}_1\orth\vect{u}_4$, and $\vect{u}_2\orth\vect{u}_3$. None
    of the other pairs of vectors are orthogonal.
  \end{solution}
\end{exercise}

\begin{exercise}
  On $C[-1,1]$, which of the following functions are orthogonal to each other?
  \begin{equation*}
    f_1(x) = x,\quad
    f_2(x) = x^2,\quad
    f_3(x) = x^3-x,\quad
    f_4(x) = 1-x^4.
  \end{equation*}
  \begin{solution}
    $f_1\orth f_2$, $f_1\orth f_4$, $f_2\orth f_3$, and $f_3\orth f_4$.
  \end{solution}
\end{exercise}

\begin{exercise}
  Consider the inner product space $\Poly_3$ of polynomials of degree
  at most $3$, with the inner product defined by
  \begin{equation*}
    \iprod{f,g} = \int_{-1}^{1} f(x)g(x)\,dx.
  \end{equation*}
  \begin{enumerate}
  \item Find the orthogonal complement of $\set{x^2,x}$.
  \item Find the orthogonal complement of $\set{x+1}$.
  \end{enumerate}
  \begin{solution}
    \begin{enumerate}
    \item Let $p(x)=ax^3+bx^2+cx+d$. Then
      $\iprod{p(x), x^2} = \frac{2}{5}b + \frac{2}{3}d$ and
      $\iprod{p(x), x} = \frac{2}{5}a + \frac{2}{3}c$.  Therefore,
      $p(x)$ is in the orthogonal complement of $\set{x^2,x}$ if and
      only if $\frac{2}{5}b + \frac{2}{3}d=0$ and
      $\frac{2}{5}a + \frac{2}{3}c=0$. It follows that
      $b=-\frac{5}{3}d$ and $a=-\frac{5}{3}c$. The general solution is
      $p(x) = -\frac{5}{3}cx^3 - \frac{5}{3}dx^2 + cx + d$. A basis
      for the orthogonal complement is
      $\set{-\frac{5}{3} x^3 + x, -\frac{5}{3} x^2 + 1}$.
    \item A basis for the orthogonal complement of $\set{x+1}$ is
      $\set{5x^3-1, 3x^2-1, 3x-1}$.      
    \end{enumerate}
  \end{solution}
\end{exercise}

\begin{exercise}
  Consider $\R^3$ as an inner product space with the usual dot
  product.  For each of the following bases of $\R^3$, state whether
  it is orthonormal, orthogonal, or neither.
  \begin{enumerate}
  \item $\set{
      \startmat{r} 1 \\ 0 \\ 0 \stopmat,
      \startmat{r} 0 \\ 1 \\ 0 \stopmat,
      \startmat{r} 0 \\ 0 \\ 1 \stopmat
    }$.
  \item $\set{
      \startmat{r} 1 \\ 0 \\ 1 \stopmat,
      \startmat{r} 0 \\ 1 \\ 1 \stopmat,
      \startmat{r} 0 \\ 0 \\ 1 \stopmat
    }$.
  \item $\set{
      \startmat{r}  1 \\ 0 \\ 2 \stopmat,
      \startmat{r}  0 \\ 1 \\ 0 \stopmat,
      \startmat{r} -2 \\ 0 \\ 1 \stopmat
    }$.
  \item $\def\arraystretch{1.2}
    \set{
      \startmat{r} \frac{3}{5} \\ \frac{4}{5} \\ 0 \stopmat,
      \startmat{r} 0 \\ 0 \\ -1 \stopmat,
      \startmat{r} \frac{4}{5} \\ -\frac{3}{5} \\ 0 \stopmat
    }$.
  \end{enumerate}
  \begin{solution}
    (a) Orthonormal (therefore also orthogonal). (b) Neither
    orthogonal nor orthonormal. (c) Orthogonal (not orthonormal). (d)
    Orthonormal (therefore also orthogonal).
  \end{solution}
\end{exercise}

\begin{exercise}
  Suppose $B=\set{\vect{u}_1,\vect{u}_2,\vect{u}_3}$ is an orthogonal
  basis for an inner product space $V$, such that
  $\norm{\vect{u}_1}=2$, $\norm{\vect{u}_2}=\sqrt{3}$, and
  $\norm{\vect{u}_3}=\sqrt{5}$.  Moreover, suppose that
  $\vect{v}\in V$ is a vector such that
  $\iprod{\vect{v},\vect{u}_1} = 1$,
  $\iprod{\vect{v},\vect{u}_2} = 2$, and
  $\iprod{\vect{v},\vect{u}_3} = -4$.  Find the coordinates of
  $\vect{v}$ with respect to $B$.
  \begin{solution}
    $\vect{v} = \frac{1}{4}\vect{u}_1 + \frac{2}{3}\vect{u}_2 - \frac{4}{5}\vect{u}_3$.
  \end{solution}
\end{exercise}

\begin{exercise}
  Suppose $B=\set{\vect{u}_1,\vect{u}_2,\vect{u}_3}$ is an
  orthogonal basis of $\R^3$. We have been told that
  \begin{equation*}
    \vect{u}_1 = \startmat{c} 1 \\ 1 \\ 0 \stopmat,
  \end{equation*}
  but it is not known what $\vect{u}_2$ and $\vect{u}_3$ are. Find the
  first coordinate of the vector
  \begin{equation*}
    \vect{v} = \startmat{c} 1 \\ 0 \\ 2 \stopmat
  \end{equation*}
  with respect to the basis $B$.
  \begin{solution}
    We have $\vect{v} = a_1\vect{u}_1 + a_2\vect{u}_2 + a_3\vect{u}_3$
    where
    $a_1 =
    \frac{\iprod{\vect{u}_1,\vect{v}}}{\iprod{\vect{u}_1,\vect{u}_1}}
    = \frac{1}{2}$. So the first coordinate is $\frac{1}{2}$.
  \end{solution}
\end{exercise}


\end{document}