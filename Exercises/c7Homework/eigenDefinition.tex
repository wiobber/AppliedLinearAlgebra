\documentclass{ximera}
\graphicspath{  %% When looking for images,
{./}            %% look here first,
{./pictures/}   %% then look for a pictures folder,
{../pictures/}  %% which may be a directory up.
{../../pictures/}  %% which may be a directory up.
{../../../pictures/}  %% which may be a directory up.
{../../../../pictures/}  %% which may be a directory up.
}

\usepackage{listings}
\usepackage{circuitikz}
\usepackage{xcolor}
\usepackage{amsmath,amsthm}
\usepackage{subcaption}
\usepackage{graphicx}
\usepackage{tikz}
\usepackage{tikz-3dplot}
\usepackage{amsfonts}
\usepackage{mdframed} % For framing content
\usepackage{tikz-cd}

  \renewcommand{\vector}[1]{\left\langle #1\right\rangle}
  \newcommand{\arrowvec}[1]{{\overset{\rightharpoonup}{#1}}}
  \newcommand{\ro}{\texttt{R}}%% row operation
  \newcommand{\dotp}{\bullet}%% dot product
  \renewcommand{\l}{\ell}
  \let\defaultAnswerFormat\answerFormatBoxed
  \usetikzlibrary{calc,bending}
  \tikzset{>=stealth}
  




%make a maroon color
\definecolor{maroon}{RGB}{128,0,0}
%make a dark blue color
\definecolor{darkblue}{RGB}{0,0,139}
%define the color fourier0 to be the maroon color
\definecolor{fourier0}{RGB}{128,0,0}
%define the color fourier1 to be the dark blue color
\definecolor{fourier1}{RGB}{0,0,139}
%define the color fourier 1t to be the light blue color
\definecolor{fourier1t}{RGB}{173,216,230}
%define the color fourier2 to be the dark green color
\definecolor{fourier2}{RGB}{0,100,0}
%define teh color fourier2t to be the light green color
\definecolor{fourier2t}{RGB}{144,238,144}
%define the color fourier3 to be the dark purple color
\definecolor{fourier3}{RGB}{128,0,128}
%define the color fourier3t to be the light purple color
\definecolor{fourier3t}{RGB}{221,160,221}
%define the color fourier0t to be the red color
\definecolor{fourier0t}{RGB}{255,0,0}
%define the color fourier4 to be the orange color
\definecolor{fourier4}{RGB}{255,165,0}
%define the color fourier4t to be the darker orange color
\definecolor{fourier4t}{RGB}{255,215,0}
%define the color fourier5 to be the yellow color
\definecolor{fourier5}{RGB}{255,255,0}
%define the color fourier5t to be the darker yellow color
\definecolor{fourier5t}{RGB}{255,255,100}
%define the color fourier6 to be the green color
\definecolor{fourier6}{RGB}{0,128,0}
%define the color fourier6t to be the darker green color
\definecolor{fourier6t}{RGB}{0,255,0}

%New commands for this doc for errors in copying
\newcommand{\eigenvar}{\lambda}
%\newcommand{\vect}[1]{\mathbf{#1}}
\renewcommand{\th}{^{\text{th}}}
\newcommand{\st}{^{\text{st}}}
\newcommand{\nd}{^{\text{nd}}}
\newcommand{\rd}{^{\text{rd}}}
\newcommand{\paren}[1]{\left(#1\right)}
\newcommand{\abs}[1]{\left|#1\right|}
\newcommand{\R}{\mathbb{R}}
\newcommand{\C}{\mathbb{C}}
\newcommand{\Hilb}{\mathbb{H}}
\newcommand{\qq}[1]{\text{#1}}
\newcommand{\Z}{\mathbb{Z}}
\newcommand{\N}{\mathbb{N}}
\newcommand{\q}[1]{\text{``#1''}}
%\newcommand{\mat}[1]{\begin{bmatrix}#1\end{bmatrix}}
\newcommand{\rref}{\text{reduced row echelon form}}
\newcommand{\ef}{\text{echelon form}}
\newcommand{\ohm}{\Omega}
\newcommand{\volt}{\text{V}}
\newcommand{\amp}{\text{A}}
\newcommand{\Seq}{\textbf{Seq}}
\newcommand{\Poly}{\textbf{P}}
\renewcommand{\quad}{\text{    }}
\newcommand{\roweq}{\simeq}
\newcommand{\rowop}{\simeq}
\newcommand{\rowswap}{\leftrightarrow}
\newcommand{\Mat}{\textbf{M}}
\newcommand{\Func}{\textbf{Func}}
\newcommand{\Hw}{\textbf{Hamming weight}}
\newcommand{\Hd}{\textbf{Hamming distance}}
\newcommand{\rank}{\text{rank}}
\newcommand{\longvect}[1]{\overrightarrow{#1}}
% Define the circled command
\newcommand{\circled}[1]{%
  \tikz[baseline=(char.base)]{
    \node[shape=circle,draw,inner sep=2pt,red,fill=red!20,text=black] (char) {#1};}%
}

% Define custom command \strikeh that just puts red text on the 2nd argument
\newcommand{\strikeh}[2]{\textcolor{red}{#2}}

% Define custom command \strikev that just puts red text on the 2nd argument
\newcommand{\strikev}[2]{\textcolor{red}{#2}}

%more new commands for this doc for errors in copying
\newcommand{\SI}{\text{SI}}
\newcommand{\kg}{\text{kg}}
\newcommand{\m}{\text{m}}
\newcommand{\s}{\text{s}}
\newcommand{\norm}[1]{\left\|#1\right\|}
\newcommand{\col}{\text{col}}
\newcommand{\sspan}{\text{span}}
\newcommand{\proj}{\text{proj}}
\newcommand{\set}[1]{\left\{#1\right\}}
\newcommand{\degC}{^\circ\text{C}}
\newcommand{\centroid}[1]{\overline{#1}}
\newcommand{\dotprod}{\boldsymbol{\cdot}}
%\newcommand{\coord}[1]{\begin{bmatrix}#1\end{bmatrix}}
\newcommand{\iprod}[1]{\langle #1 \rangle}
\newcommand{\adjoint}{^{*}}
\newcommand{\conjugate}[1]{\overline{#1}}
\newcommand{\eigenvarA}{\lambda}
\newcommand{\eigenvarB}{\mu}
\newcommand{\orth}{\perp}
\newcommand{\bigbracket}[1]{\left[#1\right]}
\newcommand{\textiff}{\text{ if and only if }}
\newcommand{\adj}{\text{adj}}
\newcommand{\ijth}{\emph{ij}^\text{th}}
\newcommand{\minor}[2]{M_{#2}}
\newcommand{\cofactor}{\text{C}}
\newcommand{\shift}{\textbf{shift}}
\newcommand{\startmat}[1]{
  \left[\begin{array}{#1}
}
\newcommand{\stopmat}{\end{array}\right]}
%a command to give a name to explorations and hints and theorems
\newcommand{\name}[1]{\begin{centering}\textbf{#1}\end{centering}}
\newcommand{\vect}[1]{\vec{#1}}
\newcommand{\dfn}[1]{\textbf{#1}}
\newcommand{\transpose}{\mathsf{T}}
\newcommand{\mtlb}[2][black]{\texttt{\textcolor{#1}{#2}}}
\newcommand{\RR}{\mathbb{R}} % Real numbers
\newcommand{\id}{\text{id}}

\author{Zack Reed}
%borrowed from selinger linear algebra
\begin{document}


\begin{exercise}
  Consider the matrix
  \begin{equation*}
    A = \startmat{rrr}
      1 & -2 & -2 \\
      2 & -3 & -2 \\
      -2 & 2 &  1 \\
    \stopmat.
  \end{equation*}
  Which of the following vectors are eigenvectors of $A$? Find the
  corresponding eigenvalues.
  \begin{equation*}
    \vect{v}_1 = \startmat{r} 1 \\ 1 \\ -1 \stopmat,\quad
    \vect{v}_2 = \startmat{r} 1 \\ 1 \\ 0 \stopmat,\quad
    \vect{v}_3 = \startmat{r} 2 \\ 2 \\ -1 \stopmat,\quad
    \vect{v}_4 = \startmat{r} 0 \\ -1 \\ 1 \stopmat.
  \end{equation*}
\end{exercise}

\begin{exercise}
  Let
  \begin{equation*}
    A = \startmat{rrr}
      1  &  0 & 0 \\
      -5 & -1 & 5 \\
      -3 &  0 & 4 \\
    \stopmat.
  \end{equation*}
  Find the eigenvectors corresponding to the eigenvalue $\eigenvar=4$.
\end{exercise}

\begin{exercise}
  Let
  \begin{equation*}
    A = \startmat{rrr}
      7 &  -4 &   8 \\
      -1 &  4 &  -2 \\
      -2 &  2 &  -1 \\
    \stopmat.
  \end{equation*}
  Find the eigenvectors corresponding to the eigenvalue $\eigenvar=3$.
\end{exercise}

\begin{exercise}
  Let
  \begin{equation*}
    A = \startmat{rrr}
      4 &   0 &   3 \\
      -3 &   1 &  -3 \\
      0 &   0 &   1 \\
    \stopmat.
  \end{equation*}
  This matrix has eigenvalues $\eigenvar=1$ and $\eigenvar=4$. Find a
  basis for each eigenspace.
\end{exercise}

\begin{exercise}
  Let
  \begin{equation*}
    A = \startmat{rrr}
      2 &   4 &  -4 \\
      -1 &  6 &  -9 \\
      0 &   0 &  -3 \\
    \stopmat.
  \end{equation*}
  This matrix has eigenvalues $\eigenvar=-3$ and $\eigenvar=4$. Find a
  basis for each eigenspace.
\end{exercise}

\begin{exercise}
  Suppose $A$ is a $3\times 3$-matrix with eigenvalues
  $\eigenvar_1=1$, $\eigenvar_2=0$, and $\eigenvar_3=2$ and
  corresponding eigenvectors
  \begin{equation*}
    \vect{v}_1 = \startmat{r}
      -1 \\
      -2 \\
      -2
    \stopmat,
    \quad
    \vect{v}_2 = \startmat{r}
      1 \\
      1 \\
      1
    \stopmat,
    \quad\mbox{and}\quad
    \vect{v}_3 = \startmat{r}
      -1 \\
      -4 \\
      -3
    \stopmat.
  \end{equation*}
  (By ``corresponding'', we mean that $\vect{v}_1$ corresponds to
  $\eigenvar_1$, $\vect{v}_2$ corresponds to $\eigenvar_2$, and so
  on).  Find
  \begin{equation*}
    A\startmat{r}
      3 \\
      -4 \\
      3
    \stopmat.
  \end{equation*}
  % \begin{solution}
  % \end{solution}
\end{exercise}

\begin{exercise}
  Let $A$ be an $n\times n$-matrix, and assume $\eigenvar$ is an
  eigenvalue of $A$. Show that $\eigenvar^2$ is an eigenvalue of
  $A^2$.
  \begin{solution}
    If $\vect{v}$ is an eigenvector corresponding to the eigenvalue
    $\eigenvar$, then $A^2\vect{v}=A(A\vect{v}) =
    A(\eigenvar\vect{v}) = \eigenvar(A\vect{v}) =
    \eigenvar^2\vect{v}$. Therefore, $\vect{v}$ is an eigenvector of
    $A^2$ with eigenvalue $\eigenvar^2$.
  \end{solution}
\end{exercise}

\begin{exercise}
  Let $A$ be an invertible $n\times n$-matrix, and assume $\eigenvar$
  is an eigenvalue of $A$. Show that $\eigenvar\neq 0$ and that
  $\eigenvar^{-1}$ is an eigenvalue of $A^{-1}$.
  \begin{solution}
    We have
    $\eigenvar A^{-1}\vect{v} = A^{-1}\eigenvar\vect{v} =
    A^{-1}A\vect{v} = \vect{v}$. Since $\vect{v}\neq 0$, this implies
    $\eigenvar\neq 0$. Moreover, it implies
    $A^{-1}\vect{v} = \eigenvar^{-1}\vect{v}$. Thus, $\eigenvar^{-1}$
    is an eigenvalue of $A^{-1}$.
  \end{solution}
\end{exercise}

\begin{exercise}
  If $A$ is an $n\times n$-matrix and $c$ is a non-zero constant,
  compare the eigenvalues of $A$ and $cA$.
  \begin{solution}
    Say $A\vect{v}=\eigenvar \vect{v}$. Then
    $cA\vect{v}=c\eigenvar \vect{v}$ and so the eigenvalues of $cA$ are
    just $c\eigenvar$ where $\eigenvar$ is an eigenvalue of $A$.
  \end{solution}
\end{exercise}

\begin{exercise}
  Let $A,B$ be invertible $n\times n$-matrices which commute. That is,
  $AB=BA$. Suppose $\vect{v}$ is an eigenvector of $B$. Show that then
  $A\vect{v}$ must also be an eigenvector for $B$.
  \begin{solution}
    Suppose $\vect{v}$ is an eigenvector of $B$, i.e.,
    $B\vect{v}=\eigenvar \vect{v}$. Then
    $BA\vect{v}=AB\vect{v} =A\eigenvar \vect{v}=\eigenvar A\vect{v}$,
    and therefore $A\vect{v}$ is an eigenvector of $B$.
  \end{solution}
\end{exercise}

\begin{exercise}
  Suppose $A$ is an $n\times n$-matrix and it satisfies $A^m=A$ for
  some $m$ a positive integer larger than 1. Show that if $\eigenvar$
  is an eigenvalue of $A$ then $\eigenvar$ equals either $0$, $1$, or
  $-1$.
  \begin{solution}
    Let $\vect{v}$ be the eigenvector. Then
    $A^m\vect{v}=\eigenvar^m\vect{v}$ and
    $A^m\vect{v}=A\vect{v}=\eigenvar\vect{v}$. Therefore
    $\eigenvar^m=\eigenvar$. Hence if $\eigenvar \neq 0$, we must
    have $\eigenvar^{m-1}=1$, which implies that $\eigenvar=\pm 1$.
  \end{solution}
\end{exercise}

\begin{exercise}
  Show that if $A\vect{v}=\eigenvar \vect{v}$ and
  $A\vect{w}=\eigenvar \vect{w}$, then whenever $k,p$ are scalars,
  \begin{equation*}
    A(k\vect{v}+p\vect{w}) =\eigenvar (k\vect{v}+p\vect{w})
  \end{equation*}
  Does this imply that $k\vect{v}+p\vect{w}$ is an eigenvector? Explain.
  \begin{solution}
    The formula follows from properties of matrix
    multiplication. However, this vector might not be an eigenvector
    because it might equal $0$ and eigenvectors cannot equal $0$.
  \end{solution}
\end{exercise}

\end{document}