\documentclass{ximera}
\graphicspath{  %% When looking for images,
{./}            %% look here first,
{./pictures/}   %% then look for a pictures folder,
{../pictures/}  %% which may be a directory up.
{../../pictures/}  %% which may be a directory up.
{../../../pictures/}  %% which may be a directory up.
{../../../../pictures/}  %% which may be a directory up.
}

\usepackage{listings}
\usepackage{circuitikz}
\usepackage{xcolor}
\usepackage{amsmath,amsthm}
\usepackage{subcaption}
\usepackage{graphicx}
\usepackage{tikz}
\usepackage{tikz-3dplot}
\usepackage{amsfonts}
\usepackage{mdframed} % For framing content
\usepackage{tikz-cd}

  \renewcommand{\vector}[1]{\left\langle #1\right\rangle}
  \newcommand{\arrowvec}[1]{{\overset{\rightharpoonup}{#1}}}
  \newcommand{\ro}{\texttt{R}}%% row operation
  \newcommand{\dotp}{\bullet}%% dot product
  \renewcommand{\l}{\ell}
  \let\defaultAnswerFormat\answerFormatBoxed
  \usetikzlibrary{calc,bending}
  \tikzset{>=stealth}
  




%make a maroon color
\definecolor{maroon}{RGB}{128,0,0}
%make a dark blue color
\definecolor{darkblue}{RGB}{0,0,139}
%define the color fourier0 to be the maroon color
\definecolor{fourier0}{RGB}{128,0,0}
%define the color fourier1 to be the dark blue color
\definecolor{fourier1}{RGB}{0,0,139}
%define the color fourier 1t to be the light blue color
\definecolor{fourier1t}{RGB}{173,216,230}
%define the color fourier2 to be the dark green color
\definecolor{fourier2}{RGB}{0,100,0}
%define teh color fourier2t to be the light green color
\definecolor{fourier2t}{RGB}{144,238,144}
%define the color fourier3 to be the dark purple color
\definecolor{fourier3}{RGB}{128,0,128}
%define the color fourier3t to be the light purple color
\definecolor{fourier3t}{RGB}{221,160,221}
%define the color fourier0t to be the red color
\definecolor{fourier0t}{RGB}{255,0,0}
%define the color fourier4 to be the orange color
\definecolor{fourier4}{RGB}{255,165,0}
%define the color fourier4t to be the darker orange color
\definecolor{fourier4t}{RGB}{255,215,0}
%define the color fourier5 to be the yellow color
\definecolor{fourier5}{RGB}{255,255,0}
%define the color fourier5t to be the darker yellow color
\definecolor{fourier5t}{RGB}{255,255,100}
%define the color fourier6 to be the green color
\definecolor{fourier6}{RGB}{0,128,0}
%define the color fourier6t to be the darker green color
\definecolor{fourier6t}{RGB}{0,255,0}

%New commands for this doc for errors in copying
\newcommand{\eigenvar}{\lambda}
%\newcommand{\vect}[1]{\mathbf{#1}}
\renewcommand{\th}{^{\text{th}}}
\newcommand{\st}{^{\text{st}}}
\newcommand{\nd}{^{\text{nd}}}
\newcommand{\rd}{^{\text{rd}}}
\newcommand{\paren}[1]{\left(#1\right)}
\newcommand{\abs}[1]{\left|#1\right|}
\newcommand{\R}{\mathbb{R}}
\newcommand{\C}{\mathbb{C}}
\newcommand{\Hilb}{\mathbb{H}}
\newcommand{\qq}[1]{\text{#1}}
\newcommand{\Z}{\mathbb{Z}}
\newcommand{\N}{\mathbb{N}}
\newcommand{\q}[1]{\text{``#1''}}
%\newcommand{\mat}[1]{\begin{bmatrix}#1\end{bmatrix}}
\newcommand{\rref}{\text{reduced row echelon form}}
\newcommand{\ef}{\text{echelon form}}
\newcommand{\ohm}{\Omega}
\newcommand{\volt}{\text{V}}
\newcommand{\amp}{\text{A}}
\newcommand{\Seq}{\textbf{Seq}}
\newcommand{\Poly}{\textbf{P}}
\renewcommand{\quad}{\text{    }}
\newcommand{\roweq}{\simeq}
\newcommand{\rowop}{\simeq}
\newcommand{\rowswap}{\leftrightarrow}
\newcommand{\Mat}{\textbf{M}}
\newcommand{\Func}{\textbf{Func}}
\newcommand{\Hw}{\textbf{Hamming weight}}
\newcommand{\Hd}{\textbf{Hamming distance}}
\newcommand{\rank}{\text{rank}}
\newcommand{\longvect}[1]{\overrightarrow{#1}}
% Define the circled command
\newcommand{\circled}[1]{%
  \tikz[baseline=(char.base)]{
    \node[shape=circle,draw,inner sep=2pt,red,fill=red!20,text=black] (char) {#1};}%
}

% Define custom command \strikeh that just puts red text on the 2nd argument
\newcommand{\strikeh}[2]{\textcolor{red}{#2}}

% Define custom command \strikev that just puts red text on the 2nd argument
\newcommand{\strikev}[2]{\textcolor{red}{#2}}

%more new commands for this doc for errors in copying
\newcommand{\SI}{\text{SI}}
\newcommand{\kg}{\text{kg}}
\newcommand{\m}{\text{m}}
\newcommand{\s}{\text{s}}
\newcommand{\norm}[1]{\left\|#1\right\|}
\newcommand{\col}{\text{col}}
\newcommand{\sspan}{\text{span}}
\newcommand{\proj}{\text{proj}}
\newcommand{\set}[1]{\left\{#1\right\}}
\newcommand{\degC}{^\circ\text{C}}
\newcommand{\centroid}[1]{\overline{#1}}
\newcommand{\dotprod}{\boldsymbol{\cdot}}
%\newcommand{\coord}[1]{\begin{bmatrix}#1\end{bmatrix}}
\newcommand{\iprod}[1]{\langle #1 \rangle}
\newcommand{\adjoint}{^{*}}
\newcommand{\conjugate}[1]{\overline{#1}}
\newcommand{\eigenvarA}{\lambda}
\newcommand{\eigenvarB}{\mu}
\newcommand{\orth}{\perp}
\newcommand{\bigbracket}[1]{\left[#1\right]}
\newcommand{\textiff}{\text{ if and only if }}
\newcommand{\adj}{\text{adj}}
\newcommand{\ijth}{\emph{ij}^\text{th}}
\newcommand{\minor}[2]{M_{#2}}
\newcommand{\cofactor}{\text{C}}
\newcommand{\shift}{\textbf{shift}}
\newcommand{\startmat}[1]{
  \left[\begin{array}{#1}
}
\newcommand{\stopmat}{\end{array}\right]}
%a command to give a name to explorations and hints and theorems
\newcommand{\name}[1]{\begin{centering}\textbf{#1}\end{centering}}
\newcommand{\vect}[1]{\vec{#1}}
\newcommand{\dfn}[1]{\textbf{#1}}
\newcommand{\transpose}{\mathsf{T}}
\newcommand{\mtlb}[2][black]{\texttt{\textcolor{#1}{#2}}}
\newcommand{\RR}{\mathbb{R}} % Real numbers
\newcommand{\id}{\text{id}}

\author{Zack Reed}
%borrowed from selinger linear algebra
\begin{document}

\begin{exercise}
  Find the characteristic polynomial of the matrix
  \begin{equation*}
    \startmat{rr}
      3 &  -2 \\
      1 &   0 \\
    \stopmat.
  \end{equation*}
  Use the quadratic formula to find the eigenvalues.
  % \begin{solution}
  % \end{solution}
\end{exercise}

\begin{exercise}
  Find the characteristic polynomial, eigenvalues, and basic
  eigenvectors of the matrix
  \begin{equation*}
    \startmat{rr}
      9 &  10 \\
      -5 &  -6 \\
    \stopmat.
  \end{equation*}
\end{exercise}

\begin{exercise}
  Find the characteristic polynomial, eigenvalues, and basic
  eigenvectors of the matrix
  \begin{equation*}
    \startmat{rrr}
      0 &   3 &  -1 \\
      -2 &   4 &  -2 \\
      2 &  -3 &   3 \\
    \stopmat.
  \end{equation*}
  One eigenvalue is $1$.
\end{exercise}

\begin{exercise}
  Find the characteristic polynomial, eigenvalues, and basic
  eigenvectors of the matrix
  \begin{equation*}
    \startmat{rrr}
      3 &   0 &  -2 \\
      -2 &   1 &   2 \\
      0 &   0 &   1 \\
    \stopmat.
  \end{equation*}
  One eigenvalue is $3$.
  % \begin{solution}
  % \end{solution}
\end{exercise}

\begin{exercise}
  Find the characteristic polynomial, eigenvalues, and basic
  eigenvectors of the matrix
  \begin{equation*}
    \startmat{rrr}
      9 & 2 & 8 \\
      2 & -6 & -2 \\
      -8 & 2 & -5 \\
    \stopmat.
  \end{equation*}
  One eigenvalue is $-3$.
  % \begin{solution}
  % \end{solution}
\end{exercise}

\begin{exercise}
  Which of the following matrices have no real eigenvalue?
  \begin{equation*}
    A = \startmat{rr}
      1 & 1 \\
      1 & -1 \\
    \stopmat,
    \quad
    B = \startmat{rr}
      1 & -1 \\
      1 & 1 \\
    \stopmat,
    \quad
    C = \startmat{rr}
      0 & 1 \\
      1 & 0 \\
    \stopmat.
  \end{equation*}
\end{exercise}

\begin{exercise}
  Find the eigenvalues and eigenvectors of the following triangular
  matrix:
  \begin{equation*}
    \startmat{rrr}
      3 & 2 & 2 \\
      0 & 1 & -2 \\
      0 & 0 & -1 \\
    \stopmat.
  \end{equation*}
\end{exercise}

\begin{exercise}
  Is it possible for a non-zero matrix to have only $0$ as an eigenvalue?
  \vspace{1mm}
  \begin{solution}
    Yes. $\startmat{cc}
      0 & 1 \\
      0 & 0
    \stopmat$ works.
  \end{solution}
\end{exercise}

\section*{Exercises}

\begin{exercise}
  Let
  \begin{equation*}
    A = \startmat{rr}
      5 & 7 \\
      -4 & 3 \\
    \stopmat.
  \end{equation*}
  Find the characteristic polynomial $p(\eigenvar)$, and compute
  $p(A)$.
\end{exercise}

\begin{exercise}
  Let
  \begin{equation*}
    A = \startmat{rrr}
      1 & 2 & 0 \\
      0 & 2 & -1 \\
      0 & 1 & 4 \\
    \stopmat.
  \end{equation*}
  Find the characteristic polynomial $p(\eigenvar)$, and compute
  $p(A)$.
\end{exercise}

\begin{exercise}
  \begin{enumerate}
  \item Let $A$ be a $2\times 2$-matrix. Prove that $A^2$ is a linear
    combination of $A$ and $I$.
  \item Give an example of a $3\times 3$-matrix $A$ such that $A^2$
    is not a linear combination of $A$ and $I$.
  \end{enumerate}
  \begin{solution}
    \begin{enumerate}
    \item The characteristic polynomial of $A$ is a quadratic
      polynomial, and therefore it is of the form
      $p(\eigenvar) = \eigenvar^2 + b\eigenvar + c$, for some
      $r,s\in\R$. By the Cayley-Hamilton theorem, $p(A)=0$, therefore
      $A^2 = -bA - cI$. This proves that $A^2$ is a linear combination
      of $A$ and $I$.
    \item $A=\startmat{rrr}
        0 & 1 & 0 \\
        0 & 0 & 1 \\
        0 & 0 & 0 \\
      \stopmat$.
    \end{enumerate}
  \end{solution}
\end{exercise}


\end{document}