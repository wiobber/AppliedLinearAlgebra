\documentclass{ximera}
\graphicspath{  %% When looking for images,
{./}            %% look here first,
{./pictures/}   %% then look for a pictures folder,
{../pictures/}  %% which may be a directory up.
{../../pictures/}  %% which may be a directory up.
{../../../pictures/}  %% which may be a directory up.
{../../../../pictures/}  %% which may be a directory up.
}

\usepackage{listings}
\usepackage{circuitikz}
\usepackage{xcolor}
\usepackage{amsmath,amsthm}
\usepackage{subcaption}
\usepackage{graphicx}
\usepackage{tikz}
\usepackage{tikz-3dplot}
\usepackage{amsfonts}
\usepackage{mdframed} % For framing content
\usepackage{tikz-cd}

  \renewcommand{\vector}[1]{\left\langle #1\right\rangle}
  \newcommand{\arrowvec}[1]{{\overset{\rightharpoonup}{#1}}}
  \newcommand{\ro}{\texttt{R}}%% row operation
  \newcommand{\dotp}{\bullet}%% dot product
  \renewcommand{\l}{\ell}
  \let\defaultAnswerFormat\answerFormatBoxed
  \usetikzlibrary{calc,bending}
  \tikzset{>=stealth}
  




%make a maroon color
\definecolor{maroon}{RGB}{128,0,0}
%make a dark blue color
\definecolor{darkblue}{RGB}{0,0,139}
%define the color fourier0 to be the maroon color
\definecolor{fourier0}{RGB}{128,0,0}
%define the color fourier1 to be the dark blue color
\definecolor{fourier1}{RGB}{0,0,139}
%define the color fourier 1t to be the light blue color
\definecolor{fourier1t}{RGB}{173,216,230}
%define the color fourier2 to be the dark green color
\definecolor{fourier2}{RGB}{0,100,0}
%define teh color fourier2t to be the light green color
\definecolor{fourier2t}{RGB}{144,238,144}
%define the color fourier3 to be the dark purple color
\definecolor{fourier3}{RGB}{128,0,128}
%define the color fourier3t to be the light purple color
\definecolor{fourier3t}{RGB}{221,160,221}
%define the color fourier0t to be the red color
\definecolor{fourier0t}{RGB}{255,0,0}
%define the color fourier4 to be the orange color
\definecolor{fourier4}{RGB}{255,165,0}
%define the color fourier4t to be the darker orange color
\definecolor{fourier4t}{RGB}{255,215,0}
%define the color fourier5 to be the yellow color
\definecolor{fourier5}{RGB}{255,255,0}
%define the color fourier5t to be the darker yellow color
\definecolor{fourier5t}{RGB}{255,255,100}
%define the color fourier6 to be the green color
\definecolor{fourier6}{RGB}{0,128,0}
%define the color fourier6t to be the darker green color
\definecolor{fourier6t}{RGB}{0,255,0}

%New commands for this doc for errors in copying
\newcommand{\eigenvar}{\lambda}
%\newcommand{\vect}[1]{\mathbf{#1}}
\renewcommand{\th}{^{\text{th}}}
\newcommand{\st}{^{\text{st}}}
\newcommand{\nd}{^{\text{nd}}}
\newcommand{\rd}{^{\text{rd}}}
\newcommand{\paren}[1]{\left(#1\right)}
\newcommand{\abs}[1]{\left|#1\right|}
\newcommand{\R}{\mathbb{R}}
\newcommand{\C}{\mathbb{C}}
\newcommand{\Hilb}{\mathbb{H}}
\newcommand{\qq}[1]{\text{#1}}
\newcommand{\Z}{\mathbb{Z}}
\newcommand{\N}{\mathbb{N}}
\newcommand{\q}[1]{\text{``#1''}}
%\newcommand{\mat}[1]{\begin{bmatrix}#1\end{bmatrix}}
\newcommand{\rref}{\text{reduced row echelon form}}
\newcommand{\ef}{\text{echelon form}}
\newcommand{\ohm}{\Omega}
\newcommand{\volt}{\text{V}}
\newcommand{\amp}{\text{A}}
\newcommand{\Seq}{\textbf{Seq}}
\newcommand{\Poly}{\textbf{P}}
\renewcommand{\quad}{\text{    }}
\newcommand{\roweq}{\simeq}
\newcommand{\rowop}{\simeq}
\newcommand{\rowswap}{\leftrightarrow}
\newcommand{\Mat}{\textbf{M}}
\newcommand{\Func}{\textbf{Func}}
\newcommand{\Hw}{\textbf{Hamming weight}}
\newcommand{\Hd}{\textbf{Hamming distance}}
\newcommand{\rank}{\text{rank}}
\newcommand{\longvect}[1]{\overrightarrow{#1}}
% Define the circled command
\newcommand{\circled}[1]{%
  \tikz[baseline=(char.base)]{
    \node[shape=circle,draw,inner sep=2pt,red,fill=red!20,text=black] (char) {#1};}%
}

% Define custom command \strikeh that just puts red text on the 2nd argument
\newcommand{\strikeh}[2]{\textcolor{red}{#2}}

% Define custom command \strikev that just puts red text on the 2nd argument
\newcommand{\strikev}[2]{\textcolor{red}{#2}}

%more new commands for this doc for errors in copying
\newcommand{\SI}{\text{SI}}
\newcommand{\kg}{\text{kg}}
\newcommand{\m}{\text{m}}
\newcommand{\s}{\text{s}}
\newcommand{\norm}[1]{\left\|#1\right\|}
\newcommand{\col}{\text{col}}
\newcommand{\sspan}{\text{span}}
\newcommand{\proj}{\text{proj}}
\newcommand{\set}[1]{\left\{#1\right\}}
\newcommand{\degC}{^\circ\text{C}}
\newcommand{\centroid}[1]{\overline{#1}}
\newcommand{\dotprod}{\boldsymbol{\cdot}}
%\newcommand{\coord}[1]{\begin{bmatrix}#1\end{bmatrix}}
\newcommand{\iprod}[1]{\langle #1 \rangle}
\newcommand{\adjoint}{^{*}}
\newcommand{\conjugate}[1]{\overline{#1}}
\newcommand{\eigenvarA}{\lambda}
\newcommand{\eigenvarB}{\mu}
\newcommand{\orth}{\perp}
\newcommand{\bigbracket}[1]{\left[#1\right]}
\newcommand{\textiff}{\text{ if and only if }}
\newcommand{\adj}{\text{adj}}
\newcommand{\ijth}{\emph{ij}^\text{th}}
\newcommand{\minor}[2]{M_{#2}}
\newcommand{\cofactor}{\text{C}}
\newcommand{\shift}{\textbf{shift}}
\newcommand{\startmat}[1]{
  \left[\begin{array}{#1}
}
\newcommand{\stopmat}{\end{array}\right]}
%a command to give a name to explorations and hints and theorems
\newcommand{\name}[1]{\begin{centering}\textbf{#1}\end{centering}}
\newcommand{\vect}[1]{\vec{#1}}
\newcommand{\dfn}[1]{\textbf{#1}}
\newcommand{\transpose}{\mathsf{T}}
\newcommand{\mtlb}[2][black]{\texttt{\textcolor{#1}{#2}}}
\newcommand{\RR}{\mathbb{R}} % Real numbers
\newcommand{\id}{\text{id}}

\author{Zack Reed}
%borrowed from selinger linear algebra
\begin{document}


\begin{exercise}
    
    % Matrix A
    Let matrix \( A \) be defined as follows:
    \[
    A = \startmat{rrrrrrrrrrrr}
        1 & -5 & 3 & 4 & -2 & 8 & 0 & 7 & -3 & 6 & 5 & -1 \\
        0 & 9 & 6 & 5 & -7 & 2 & 8 & -4 & 3 & 2 & -5 & 1 \\
        7 & 8 & -1 & 6 & 3 & -9 & 2 & -8 & 4 & 0 & 5 & 6 \\
        2 & -2 & 0 & 3 & 7 & -4 & -5 & 1 & -6 & 8 & -3 & 9 \\
        5 & 4 & -8 & -7 & 2 & 3 & 9 & 1 & -5 & 6 & -2 & 0 \\
        -6 & 3 & 9 & 8 & 4 & 0 & -7 & 5 & 6 & -1 & 2 & 3 \\
        0 & 7 & 2 & -5 & 8 & 9 & -3 & 4 & -1 & 6 & -4 & 7 \\
        1 & -6 & 5 & 2 & 0 & -8 & 4 & -7 & 3 & 5 & 9 & -3 \\
        9 & 4 & 3 & 1 & -5 & 6 & -7 & 0 & 8 & 2 & -6 & 1 \\
        -8 & 2 & -1 & 0 & 5 & 7 & 9 & 3 & -4 & 6 & 8 & -2 \\
        4 & 5 & -6 & 3 & -7 & 0 & 8 & 2 & 1 & -4 & 6 & 9 \\
        7 & -3 & 0 & 9 & -6 & 5 & 2 & 1 & 8 & 4 & 0 & -5
    \stopmat
    \]
    
    \begin{hint}
    The MATLAB code for entering matrix \( A \):
    \begin{verbatim}
    A = [
        1 -5 3 4 -2 8 0 7 -3 6 5 -1;
        0 9 6 5 -7 2 8 -4 3 2 -5 1;
        7 8 -1 6 3 -9 2 -8 4 0 5 6;
        2 -2 0 3 7 -4 -5 1 -6 8 -3 9;
        5 4 -8 -7 2 3 9 1 -5 6 -2 0;
        -6 3 9 8 4 0 -7 5 6 -1 2 3;
        0 7 2 -5 8 9 -3 4 -1 6 -4 7;
        1 -6 5 2 0 -8 4 -7 3 5 9 -3;
        9 4 3 1 -5 6 -7 0 8 2 -6 1;
        -8 2 -1 0 5 7 9 3 -4 6 8 -2;
        4 5 -6 3 -7 0 8 2 1 -4 6 9;
        7 -3 0 9 -6 5 2 1 8 4 0 -5
    ];
    \end{verbatim}

    \end{hint}
    
    % Matrix B
    Let matrix \( B \) be defined as follows:
    \[
    B = \startmat{rrrrrrrrrrrr}
        -4 & 7 & 1 & 5 & -2 & 9 & 0 & 3 & 6 & -8 & 4 & 2 \\
        8 & 2 & -9 & 6 & 0 & 4 & -1 & 7 & -3 & 5 & -6 & 9 \\
        3 & 4 & 8 & 1 & -7 & 5 & 9 & -2 & 0 & 6 & -1 & -5 \\
        -2 & 1 & 0 & 9 & 8 & -4 & 3 & 7 & 2 & 5 & -3 & 6 \\
        9 & -6 & 4 & 0 & 7 & -8 & 1 & 3 & -5 & 6 & 2 & -1 \\
        5 & 3 & -8 & 7 & 2 & 6 & 9 & -4 & 0 & 1 & 5 & -9 \\
        -7 & 9 & 0 & 4 & 6 & 3 & 8 & -1 & 7 & -5 & 2 & 6 \\
        1 & -3 & 5 & -7 & 2 & 9 & 6 & 0 & -2 & 8 & 3 & 1 \\
        6 & 4 & -2 & 3 & 7 & -1 & 9 & 8 & 5 & 0 & 2 & -6 \\
        0 & -8 & 9 & 2 & 5 & 4 & -3 & 1 & 7 & 6 & 8 & -2 \\
        2 & -5 & 6 & 9 & 4 & -3 & 0 & 7 & 1 & 5 & -9 & 8 \\
        8 & 3 & -4 & 0 & 9 & 6 & -5 & 2 & 4 & -7 & 1 & 3
    \stopmat
    \]
    
    \begin{hint}
    The MATLAB code for entering matrix \( B \):
    \begin{verbatim}
    B = [
        -4 7 1 5 -2 9 0 3 6 -8 4 2;
        8 2 -9 6 0 4 -1 7 -3 5 -6 9;
        3 4 8 1 -7 5 9 -2 0 6 -1 -5;
        -2 1 0 9 8 -4 3 7 2 5 -3 6;
        9 -6 4 0 7 -8 1 3 -5 6 2 -1;
        5 3 -8 7 2 6 9 -4 0 1 5 -9;
        -7 9 0 4 6 3 8 -1 7 -5 2 6;
        1 -3 5 -7 2 9 6 0 -2 8 3 1;
        6 4 -2 3 7 -1 9 8 5 0 2 -6;
        0 -8 9 2 5 4 -3 1 7 6 8 -2;
        2 -5 6 9 4 -3 0 7 1 5 -9 8;
        8 3 -4 0 9 6 -5 2 4 -7 1 3
    ];
    \end{verbatim}
    \end{hint}

The provided hints give the MATLAB code for entering matrices \( A \) and \( B \).

Compute the following entries of the given matrix products:

%list various products and powers and ask for random entries. The answers are of the form $M(i,j)=\answer{number}$

\begin{enumerate}

\item $A^2(3,5)=\answer{-136}$
\item $B^2(4,6)=\answer{-3}$
\item $AB(5,8)=\answer{-53}$
\item $BA(7,9)=\answer{127}$
\item $A^3B^2(2,3)=\answer{78392}$
\item $B^3A^2(6,7)=\answer{726074}$
\item $A^2B^3(8,9)=\answer{-298394}$

\end{enumerate}
    

\end{exercise}

\end{document}