\documentclass{ximera}
\graphicspath{  %% When looking for images,
{./}            %% look here first,
{./pictures/}   %% then look for a pictures folder,
{../pictures/}  %% which may be a directory up.
{../../pictures/}  %% which may be a directory up.
{../../../pictures/}  %% which may be a directory up.
{../../../../pictures/}  %% which may be a directory up.
}

\usepackage{listings}
\usepackage{circuitikz}
\usepackage{xcolor}
\usepackage{amsmath,amsthm}
\usepackage{subcaption}
\usepackage{graphicx}
\usepackage{tikz}
\usepackage{tikz-3dplot}
\usepackage{amsfonts}
\usepackage{mdframed} % For framing content
\usepackage{tikz-cd}

  \renewcommand{\vector}[1]{\left\langle #1\right\rangle}
  \newcommand{\arrowvec}[1]{{\overset{\rightharpoonup}{#1}}}
  \newcommand{\ro}{\texttt{R}}%% row operation
  \newcommand{\dotp}{\bullet}%% dot product
  \renewcommand{\l}{\ell}
  \let\defaultAnswerFormat\answerFormatBoxed
  \usetikzlibrary{calc,bending}
  \tikzset{>=stealth}
  




%make a maroon color
\definecolor{maroon}{RGB}{128,0,0}
%make a dark blue color
\definecolor{darkblue}{RGB}{0,0,139}
%define the color fourier0 to be the maroon color
\definecolor{fourier0}{RGB}{128,0,0}
%define the color fourier1 to be the dark blue color
\definecolor{fourier1}{RGB}{0,0,139}
%define the color fourier 1t to be the light blue color
\definecolor{fourier1t}{RGB}{173,216,230}
%define the color fourier2 to be the dark green color
\definecolor{fourier2}{RGB}{0,100,0}
%define teh color fourier2t to be the light green color
\definecolor{fourier2t}{RGB}{144,238,144}
%define the color fourier3 to be the dark purple color
\definecolor{fourier3}{RGB}{128,0,128}
%define the color fourier3t to be the light purple color
\definecolor{fourier3t}{RGB}{221,160,221}
%define the color fourier0t to be the red color
\definecolor{fourier0t}{RGB}{255,0,0}
%define the color fourier4 to be the orange color
\definecolor{fourier4}{RGB}{255,165,0}
%define the color fourier4t to be the darker orange color
\definecolor{fourier4t}{RGB}{255,215,0}
%define the color fourier5 to be the yellow color
\definecolor{fourier5}{RGB}{255,255,0}
%define the color fourier5t to be the darker yellow color
\definecolor{fourier5t}{RGB}{255,255,100}
%define the color fourier6 to be the green color
\definecolor{fourier6}{RGB}{0,128,0}
%define the color fourier6t to be the darker green color
\definecolor{fourier6t}{RGB}{0,255,0}

%New commands for this doc for errors in copying
\newcommand{\eigenvar}{\lambda}
%\newcommand{\vect}[1]{\mathbf{#1}}
\renewcommand{\th}{^{\text{th}}}
\newcommand{\st}{^{\text{st}}}
\newcommand{\nd}{^{\text{nd}}}
\newcommand{\rd}{^{\text{rd}}}
\newcommand{\paren}[1]{\left(#1\right)}
\newcommand{\abs}[1]{\left|#1\right|}
\newcommand{\R}{\mathbb{R}}
\newcommand{\C}{\mathbb{C}}
\newcommand{\Hilb}{\mathbb{H}}
\newcommand{\qq}[1]{\text{#1}}
\newcommand{\Z}{\mathbb{Z}}
\newcommand{\N}{\mathbb{N}}
\newcommand{\q}[1]{\text{``#1''}}
%\newcommand{\mat}[1]{\begin{bmatrix}#1\end{bmatrix}}
\newcommand{\rref}{\text{reduced row echelon form}}
\newcommand{\ef}{\text{echelon form}}
\newcommand{\ohm}{\Omega}
\newcommand{\volt}{\text{V}}
\newcommand{\amp}{\text{A}}
\newcommand{\Seq}{\textbf{Seq}}
\newcommand{\Poly}{\textbf{P}}
\renewcommand{\quad}{\text{    }}
\newcommand{\roweq}{\simeq}
\newcommand{\rowop}{\simeq}
\newcommand{\rowswap}{\leftrightarrow}
\newcommand{\Mat}{\textbf{M}}
\newcommand{\Func}{\textbf{Func}}
\newcommand{\Hw}{\textbf{Hamming weight}}
\newcommand{\Hd}{\textbf{Hamming distance}}
\newcommand{\rank}{\text{rank}}
\newcommand{\longvect}[1]{\overrightarrow{#1}}
% Define the circled command
\newcommand{\circled}[1]{%
  \tikz[baseline=(char.base)]{
    \node[shape=circle,draw,inner sep=2pt,red,fill=red!20,text=black] (char) {#1};}%
}

% Define custom command \strikeh that just puts red text on the 2nd argument
\newcommand{\strikeh}[2]{\textcolor{red}{#2}}

% Define custom command \strikev that just puts red text on the 2nd argument
\newcommand{\strikev}[2]{\textcolor{red}{#2}}

%more new commands for this doc for errors in copying
\newcommand{\SI}{\text{SI}}
\newcommand{\kg}{\text{kg}}
\newcommand{\m}{\text{m}}
\newcommand{\s}{\text{s}}
\newcommand{\norm}[1]{\left\|#1\right\|}
\newcommand{\col}{\text{col}}
\newcommand{\sspan}{\text{span}}
\newcommand{\proj}{\text{proj}}
\newcommand{\set}[1]{\left\{#1\right\}}
\newcommand{\degC}{^\circ\text{C}}
\newcommand{\centroid}[1]{\overline{#1}}
\newcommand{\dotprod}{\boldsymbol{\cdot}}
%\newcommand{\coord}[1]{\begin{bmatrix}#1\end{bmatrix}}
\newcommand{\iprod}[1]{\langle #1 \rangle}
\newcommand{\adjoint}{^{*}}
\newcommand{\conjugate}[1]{\overline{#1}}
\newcommand{\eigenvarA}{\lambda}
\newcommand{\eigenvarB}{\mu}
\newcommand{\orth}{\perp}
\newcommand{\bigbracket}[1]{\left[#1\right]}
\newcommand{\textiff}{\text{ if and only if }}
\newcommand{\adj}{\text{adj}}
\newcommand{\ijth}{\emph{ij}^\text{th}}
\newcommand{\minor}[2]{M_{#2}}
\newcommand{\cofactor}{\text{C}}
\newcommand{\shift}{\textbf{shift}}
\newcommand{\startmat}[1]{
  \left[\begin{array}{#1}
}
\newcommand{\stopmat}{\end{array}\right]}
%a command to give a name to explorations and hints and theorems
\newcommand{\name}[1]{\begin{centering}\textbf{#1}\end{centering}}
\newcommand{\vect}[1]{\vec{#1}}
\newcommand{\dfn}[1]{\textbf{#1}}
\newcommand{\transpose}{\mathsf{T}}
\newcommand{\mtlb}[2][black]{\texttt{\textcolor{#1}{#2}}}
\newcommand{\RR}{\mathbb{R}} % Real numbers
\newcommand{\id}{\text{id}}

\author{Zack Reed}
%borrowed from selinger linear algebra
\begin{document}

%apple_pear_classify
\begin{exercise}

The data that comprises the apple and pear fruit clusters in the GeoGebra applet have been collected into two matrices, $A$ and $P$, where each column of $A$ and $P$ is a vector representing a fruit measurement. Each row, therefore, must represent

\begin{multipleChoice}
  \choice{the measurement of a feature.}
  \choice[correct]{a feature being measured.}
  \choice{a fruit.}
\end{multipleChoice}

The matrices for $A$ and $P$ are given in the Module 1 course files, saved as \texttt{apple.mat} and \texttt{pear.mat}. Load the matrices into MATLAB and determine the dimensions of each matrix.

The matrix $A$ has dimensions $\answer{3}$ rows by $\answer{100}$ columns. The matrix $P$ has dimensions $\answer{3}$ rows by $\answer{100}$ columns.

Using the data, find the average pear measurement vector $\overline{P}$ and the average apple measurement vector $\overline{A}$. Using these averages, determine which of the following fruit should be classified as a pear, apple, or neither:

\begin{hint}[A note on answer entry]
  Enter the strings "apple" or "pear" or "neither" without quotes in your answers.
  
  The distance calculations have an error tolerance of 0.1, so if you're getting the answer wrong but think you're right, check your rounding, or check your work such as the average fruit vectors or the distance calculations. When in doubt, email your instructor!
\end{hint}

\begin{enumerate}
\item
The fruit
\begin{equation*}
  \vec{F}_1=\startmat{r}
    6.04 \\
    -.91 \\
    6.62
  \stopmat
\end{equation*}
is more likely to be a $\answer[format=string]{apple}$, with a distance of $\answer[tolerance=.1]{2.99}$ from the average apple measurement vector and a distance of $\answer[tolerance=.1]{12.92}$ from the average pear measurement vector.

\item
The fruit
\begin{equation*}
  \vec{F}_2=\startmat{r}
    -2.19 \\
    7.75 \\
    6.75
  \stopmat
\end{equation*}
is more likely to be a $\answer[format=string]{pear}$, with a distance of $\answer[tolerance=.1]{10.89}$ from the average apple measurement vector and a distance of $\answer[tolerance=.1]{4.94}$ from the average pear measurement vector.

\item
The fruit
\begin{equation*}
  \vec{F}_3=\startmat{r}
    7.1 \\
    -3.55 \\
    7.2
  \stopmat
\end{equation*}
is more likely to be a $\answer[format=string]{neither}$, with a distance of $\answer[tolerance=.1]{15.91}$ from the average apple measurement vector and a distance of $\answer[tolerance=.1]{16.19}$ from the average pear measurement vector.

\item
The fruit
\begin{equation*}
  \vec{F}_4=\startmat{r}
    -2.76 \\
    4.25 \\
    2.57
  \stopmat
\end{equation*}
is more likely to be a $\answer[format=string]{pear}$, with a distance of $\answer[tolerance=.1]{10.98}$ from the average apple measurement vector and a distance of $\answer[tolerance=.1]{3.81}$ from the average pear measurement vector.

\end{enumerate}

\begin{hint}

  Make sure you're working in a folder that includes the +mod1 course folder, then load the matrices.
  \begin{verbatim}
    load +mod1/Apple.mat
    load +mod1/Pear.mat
  \end{verbatim}

  To find the average column of a matrix in MATLAB, use the \texttt{mean} function. The second argument of mean specifies whether you want to average along the rows or columns. For example, to find the average column of matrix $A$, use the command \texttt{mean(A,2)}. For example, to find the average row of matrix $P$, use the command \texttt{mean(P,1)}.

  Warning: Double check that your vectors are all of the same dimension (i.e. 1x3 vs 3x1), as this will impact the distance calculations and difference vectors.

\end{hint}

  
\end{exercise}



\end{document}